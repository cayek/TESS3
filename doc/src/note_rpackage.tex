\documentclass[10pt,a4paper]{article}

\usepackage{amssymb}
\usepackage[english]{babel}
\usepackage[utf8]{inputenc}
\usepackage{graphicx}
\usepackage{lineno}
\usepackage{cite}
\usepackage{float}
\usepackage{ccaption}
\usepackage{caption}
\usepackage{array}
\usepackage{lscape}
\usepackage[hmargin=2cm,vmargin=2cm]{geometry}
\usepackage{fancyvrb}
\usepackage{hyperref}

\title{{\tt TESS3} reference manual \\
{\tt R} package
}

\author{
        Kevin Caye (kevin.caye@imag.fr)\\
        Olivier Fran\c cois (olivier.francois@imag.fr)\\
}

\newcommand{\bp}{\mathbf{p}}
\newcommand{\LLL}{\mathcal{L}}

%% BEGIN DOC
\begin{document}


\maketitle
\begin{center}
{\it Please, print this reference manual only if it is necessary.}
\end{center}

\vspace{.5cm}

\begin{center} {\bf Summary}
\end{center}


\vspace{.5cm}

Geography is an important determinant of genetic variation in natural 
populations, and its effects are commonly investigated by analyzing population 
genetic structure using spatial ancestry estimation programs. A common issue is 
that classical spatial ancestry estimation programs do not scale with the 
dimension of the data sets generated from modern sequencing technologies, and 
more efficient algorithms are needed to analyze genome-wide patterns of 
population genetic variation in their geographic context.

The computer program {\tt TESS3}~\cite{TESS3} implements admixture models. The program has 
functionalities similar to the previous versions of {\tt 
TESS}~\cite{chen2007bayesian,durand2009spatial}, has run-times several order 
faster than those of common Bayesian clustering programs. In addition, the 
program can be used to perform genome scans for selection based on ancestral 
allele frequency differentiation statistic, and to separate non-adaptive and 
adaptive genetic variation.

This documentation aims to help users to run the {\tt R} package {\tt tess3r} 
which implements {\tt TESS3} with some {\tt R} functions that facilitate the 
post-processing of the program outputs. 

\vspace{.5cm}


\section{Program installation} 

The installation of {\tt tess3r} {\tt R} package requires that {\tt R} is install on your computer (\url{https://www.r-project.org/}). You can install the R package directly from the github repository thanks to the package devtools. 
If you don't already have {\tt devtools} R package you can install it from CRAN. In a R session paste this command:

\begin{Verbatim}[frame = single]
install.packages("devtools")
\end{Verbatim}


\noindent Then, you can install the R package. In a R session paste this command:

\begin{Verbatim}[frame = single]
devtools::install_github("cayek/TESS3/tess3r")
\end{Verbatim}


\section{Data format}

\subsection{Input files}

The R package {\tt tess3r} handles two kind of input files, the first one recording individual 
genotype data and the second one containing the geographic coordinates of  each 
sampled individual. For organism genomes of arbitrary ploidy, the standard data 
type for {\tt tess3r} is the {\bf single nucleotide polymorphism} (SNP) type.  
The genotype matrix must be formatted in the {\bf geno} format and the 
coordinate file must be formatted in the {\bf coord} format.

Users who want to process allelic data, such as microsatellite markers or AFLPs, 
and have their data in the {\tt TESS} 2.3 format can also use {\tt tess3r}. They 
need to convert their data in the geno+coord data format, and can do this using 
the {\tt tess2tess3} function implemented in the package.


\begin{itemize}
\item {\bf geno} (example.geno)

The {\bf geno} format has one row for each SNP. Each row contains 1 character 
per individual. For diploid genomes,  0 means zero copies of the reference 
allele, 1 means one copy of the reference allele, 2 means two copies of the 
reference allele, and 9 codes for some missing data. Here is an example of a 
geno file for $n=3$ individuals and $L=4$ loci.
\\
\begin{center}
\footnotesize
\begin{Verbatim}[frame=single]
112
010
091
121
\end{Verbatim}
\end{center}


\item {\bf coord} (example.coord)

The {\bf coord} format has one row for each individual. Each row contains the 
\verb|longitude| and \verb|latitude| coordinates of each individual.
\\
\begin{center}
\footnotesize
\begin{Verbatim}[frame=single]
2.5154 5.4390
-8.4293 4.0197
1.3536 5.5852
\end{Verbatim}
\end{center}

\end{itemize}

\noindent Users having their genotype data in the {\bf ped}, {\bf ancestrymap}, 
{\bf vcf} or {\bf lfmm} format  can use the {\tt R} package {\tt LEA} to convert 
them in the {\bf geno} format~\cite{frichot2015lea}. 

\section{Run TESS3}

In a {\tt R} session the {\tt TESS3} program can be run by typing: 

\begin{Verbatim}[frame=single]
> # Main parameters:
> # input.file  is the genotype data file at .geno format
> # input.coord is the coordinates data file at .coord format
> # ploidy      is the ploidy of the species, 
> #             here we assume that the data come from haploide species
> # K           is the number of ancestral population, 
> #             here we run the algorithm for K equals from 2 to 4
> # repetition  is the number of algorithm run per parameter set.
> tess3.obj = TESS3( input.file = "genotype.geno", 
>                    input.coord = "coordinates.coord", 
>                    ploidy = 1
>                    K = 2:4,
>                    repetition = 5)
\end{Verbatim}

\noindent Then, ancestry coefficients $Q$ matrix, ancestral population allele frequencies $G$ matrix 
and the $F_{\rm ST}$ vector
can be retrieved: 

\begin{Verbatim}[frame=single]
> Q = Q(tess3.obj, K = 3, repetition = 1)
> G = G(tess3.obj, K = 3, repetition = 1)
> Fst = FST(project, K = 3, repetition = 1)
\end{Verbatim}

\noindent Other features of {\tt tess3r} package are illustrated in more complete tutorial: 
\url{https://github.com/cayek/TESS3/raw/master/doc/tess3r_tutorial.pdf}


\section{Contact}
If you need assistance, do not hesitate to send us an email (kevin.caye@imag.fr 
or olivier.francois@imag.fr). 

\bibliographystyle{plain}
\bibliography{note}

\end{document}
